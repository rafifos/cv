\documentclass[10pt, a4paper]{article}

% Packages:
\usepackage[
    ignoreheadfoot, % set margins without considering header and footer
    top=2 cm, % seperation between body and page edge from the top
    bottom=2 cm, % seperation between body and page edge from the bottom
    left=2 cm, % seperation between body and page edge from the left
    right=2 cm, % seperation between body and page edge from the right
    footskip=1.0 cm, % seperation between body and footer
    % showframe % for debugging 
]{geometry} % for adjusting page geometry
\usepackage{titlesec} % for customizing section titles
\usepackage{tabularx} % for making tables with fixed width columns
\usepackage{array} % tabularx requires this
\usepackage[dvipsnames]{xcolor} % for coloring text
\definecolor{primaryColor}{RGB}{0, 79, 144} % define primary color
\usepackage{enumitem} % for customizing lists
\usepackage{fontawesome5} % for using icons
\usepackage{amsmath} % for math
\usepackage[
    pdftitle={Rafael Julio Lemos Silva's CV},
    pdfauthor={Rafael Julio Lemos Silva},
    pdfcreator={LaTeX with RenderCV},
    colorlinks=true,
    urlcolor=primaryColor
]{hyperref} % for links, metadata and bookmarks
\usepackage[pscoord]{eso-pic} % for floating text on the page
\usepackage{calc} % for calculating lengths
\usepackage{bookmark} % for bookmarks
\usepackage{lastpage} % for getting the total number of pages
\usepackage{changepage} % for one column entries (adjustwidth environment)
\usepackage{paracol} % for two and three column entries
\usepackage{ifthen} % for conditional statements
\usepackage{needspace} % for avoiding page brake right after the section title
\usepackage{iftex} % check if engine is pdflatex, xetex or luatex

% Ensure that generate pdf is machine readable/ATS parsable:
\ifPDFTeX
    \input{glyphtounicode}
    \pdfgentounicode=1
    % \usepackage[T1]{fontenc} % this breaks sb2nov
    \usepackage[utf8]{inputenc}
    \usepackage{lmodern}
\fi

\usepackage{charter}

% Some settings:
\AtBeginEnvironment{adjustwidth}{\partopsep0pt} % remove space before adjustwidth environment
\pagestyle{empty} % no header or footer
\setcounter{secnumdepth}{0} % no section numbering
\setlength{\parindent}{0pt} % no indentation
\setlength{\topskip}{0pt} % no top skip
\setlength{\columnsep}{0cm} % set column seperation
\makeatletter
\let\ps@customFooterStyle\ps@plain % Copy the plain style to customFooterStyle
\patchcmd{\ps@customFooterStyle}{\thepage}{
    \color{gray}\textit{\small Page \thepage{} of \pageref*{LastPage}}
}{}{} % replace number by desired string
\makeatother
\pagestyle{customFooterStyle}

\titleformat{\section}{\needspace{4\baselineskip}\bfseries\large}{}{0pt}{}[\vspace{1pt}\titlerule]

\titlespacing{\section}{
    % left space:
    -1pt
}{
    % top space:
    0.3 cm
}{
    % bottom space:
    0.2 cm
} % section title spacing

\renewcommand\labelitemi{$\circ$} % custom bullet points
\newenvironment{highlights}{
    \begin{itemize}[
        topsep=0.10 cm,
        parsep=0.10 cm,
        partopsep=0pt,
        itemsep=0pt,
        leftmargin=0.4 cm + 10pt
    ]
}{
    \end{itemize}
} % new environment for highlights

\newenvironment{highlightsforbulletentries}{
    \begin{itemize}[
        topsep=0.10 cm,
        parsep=0.10 cm,
        partopsep=0pt,
        itemsep=0pt,
        leftmargin=10pt
    ]
}{
    \end{itemize}
} % new environment for highlights for bullet entries


\newenvironment{onecolentry}{
    \begin{adjustwidth}{
        0.2 cm + 0.00001 cm
    }{
        0.2 cm + 0.00001 cm
    }
}{
    \end{adjustwidth}
} % new environment for one column entries

\newenvironment{twocolentry}[2][]{
    \onecolentry
    \def\secondColumn{#2}
    \setcolumnwidth{\fill, 7 cm}
    \begin{paracol}{2}
}{
    \switchcolumn \raggedleft \secondColumn
    \end{paracol}
    \endonecolentry
} % new environment for two column entries

\newenvironment{header}{
    \setlength{\topsep}{0pt}\par\kern\topsep\centering\linespread{1.5}
}{
    \par\kern\topsep
} % new environment for the header

\newcommand{\placelastupdatedtext}{% \placetextbox{<horizontal pos>}{<vertical pos>}{<stuff>}
  \AddToShipoutPictureFG*{% Add <stuff> to current page foreground
    \put(
        \LenToUnit{\paperwidth-2 cm-0.2 cm+0.05cm},
        \LenToUnit{\paperheight-1.0 cm}
    ){\vtop{{\null}\makebox[0pt][c]{
        \small\color{gray}\textit{Last updated in November 2024}\hspace{\widthof{Last updated in November 2024}}
    }}}%
  }%
}%

% save the original href command in a new command:
\let\hrefWithoutArrow\href

% new command for external links:
\renewcommand{\href}[2]{\hrefWithoutArrow{#1}{\ifthenelse{\equal{#2}{}}{ }{#2 }\raisebox{.15ex}{\footnotesize \faExternalLink*}}}


\begin{document}
    \newcommand{\AND}{\unskip
        \cleaders\copy\ANDbox\hskip\wd\ANDbox
        \ignorespaces
    }
    \newsavebox\ANDbox
    \sbox\ANDbox{}

    \placelastupdatedtext
    \begin{header}
        \textbf{\fontsize{24 pt}{24 pt}\selectfont Rafael Julio Lemos Silva}

        \vspace{0.3 cm}

        \normalsize
        \mbox{{\color{black}\footnotesize\faMapMarker*}\hspace*{0.13cm}São Paulo, Brazil}%
        \kern 0.25 cm%
        \AND%
        \kern 0.25 cm%
        \mbox{\hrefWithoutArrow{mailto:work@rafifos.dev}{\color{black}{\footnotesize\faEnvelope[regular]}\hspace*{0.13cm}work@rafifos.dev}}%
        \kern 0.25 cm%
        \AND%
        \kern 0.25 cm%
        \mbox{\hrefWithoutArrow{tel:+55-11-99463-8955}{\color{black}{\footnotesize\faPhone*}\hspace*{0.13cm}(11) 99463-8955}}%
        \kern 0.25 cm%
        \AND%
        \kern 0.25 cm%
        \mbox{\hrefWithoutArrow{https://rafifos.dev/}{\color{black}{\footnotesize\faLink}\hspace*{0.13cm}rafifos.dev}}%
        \kern 0.25 cm%
        \AND%
        \kern 0.25 cm%
        \mbox{\hrefWithoutArrow{https://linkedin.com/in/rafifos}{\color{black}{\footnotesize\faLinkedinIn}\hspace*{0.13cm}rafifos}}%
        \kern 0.25 cm%
        \AND%
        \kern 0.25 cm%
        \mbox{\hrefWithoutArrow{https://github.com/rafifos}{\color{black}{\footnotesize\faGithub}\hspace*{0.13cm}rafifos}}%
    \end{header}

    \vspace{0.3 cm - 0.3 cm}


    \section{Summary}



        
        \begin{onecolentry}
            I'm detail-oriented and always strive to write clean, maintainable, and reusable code. I enjoy automating tasks and environments (Bash, Zsh, dotfiles, Docker, etc.).
        \end{onecolentry}

        \vspace{0.2 cm}

        \begin{onecolentry}
            Graduated in Systems Analysis and Development from Faculdade Impacta, I have experience with Ruby on Rails, JavaScript/TypeScript, and DevOps (containerizing services with Docker, CI/CD pipelines with GitLab CI/CD, GitHub Actions, and AWS CodeBuild), along with working with AWS, Kubernetes, Helm, and Terraform.
        \end{onecolentry}

        \vspace{0.2 cm}

        \begin{onecolentry}
            My main stack is JavaScript, TypeScript, Node.js, React, and Next.js.
        \end{onecolentry}

        \vspace{0.2 cm}

        \begin{onecolentry}
            I'm also active in Android OS communities, where I contribute with investigations, bug fixes, localization, and internal Android components.
        \end{onecolentry}


    
    \section{Experience}



        
        \begin{twocolentry}{
        \textit{São Paulo, Brazil}    
            
        \textit{Jul 2024 to present}}
            \textbf{Systems Development Specialist}
            
            \textit{International School Education}
        \end{twocolentry}

        \vspace{0.10 cm}
        \begin{onecolentry}
            \begin{highlights}
                \item Implementation of interactive student activities using H5P
                \item Integrations with third-party services
                \item Case studies on AI use cases
            \end{highlights}
        \end{onecolentry}


        \vspace{0.2 cm}

        \begin{twocolentry}{
        \textit{São Paulo, Brazil}    
            
        \textit{Jul 2023 to Jun 2024}}
            \textbf{Senior Systems Developer}
            
            \textit{AeC}
        \end{twocolentry}

        \vspace{0.10 cm}
        \begin{onecolentry}
            \begin{highlights}
                \item Comment moderation for courses and articles
                \item Public courses to expand the LMS's reach
                \item Bugfixes for SCORM contents
                \item Optimizations for the LMS's performance
                \item Reports for the LMS's usage
                \item Google Analytics integration
            \end{highlights}
        \end{onecolentry}


        \vspace{0.2 cm}

        \begin{twocolentry}{
        \textit{São Paulo, Brazil}    
            
        \textit{Jan 2023 to Jun 2023}}
            \textbf{Senior Systems Developer}
            
            \textit{Stefanini Brazil}
        \end{twocolentry}

        \vspace{0.10 cm}
        \begin{onecolentry}
            \begin{highlights}
                \item Development of flows such as Power Outage Reporting and Facial/Document Recognition
                \item Creation of the Digital Agency Homepage, focusing on intuitive navigation
                \item Bug fixes and improvements to ensure a seamless user experience
            \end{highlights}
        \end{onecolentry}


        \vspace{0.2 cm}

        \begin{twocolentry}{
        \textit{Porto Alegre, Brazil}    
            
        \textit{Feb 2021 to Oct 2022}}
            \textbf{Front-end Web Developer}
            
            \textit{Me Salva!}
        \end{twocolentry}

        \vspace{0.10 cm}
        \begin{onecolentry}
            \begin{highlights}
                \item Participated in architectural discussions and the development of the company's Design System (MARS), collaborating with the design consultancy across all phases, from prototyping to implementation, ensuring standardization and visual consistency
                \item Contributed to the rewrite of the Me Salva! platform using MARS, engaging in architecture, design, and feature implementation decisions, enhancing scalability and modernization
                \item Developed internal libraries to improve Developer Experience, configuring TypeScript, linters (ESLint, Prettier), Jest, Conventional Commits, Rollup, semantic-release, and CI/CD (Heroku and GitHub Actions)
                \item Developed an internal CMS allowing the content team to create and manage new pages, and content, improving the platform's agility and reducing the development team's workload
                \item Acted as a technical reference for junior developers, providing support, pair programming, and assisting in the maintenance of legacy projects with improvements and new functionalities
            \end{highlights}
        \end{onecolentry}


        \vspace{0.2 cm}

        \begin{twocolentry}{
        \textit{São Paulo, Brazil}    
            
        \textit{Aug 2019 to Feb 2021}}
            \textbf{Web Development Intern - Junior Web Developer}
            
            \textit{Quero Educação}
        \end{twocolentry}

        \vspace{0.10 cm}
        \begin{onecolentry}
            \begin{highlights}
                \item Contributed to the development of new features, product improvements, and strategic innovations at Quero, using agile methodologies and modern tools
                \item Developed critical features for the app (React Native), including Digital Enrollment and Branding updates, while also participating in React Native version upgrades to ensure best practices and performance optimization
                \item Enhanced the resilience and stability of the Digital Enrollment feature by fixing bugs and implementing improvements, delivering a smoother user experience
                \item Participated in a multidisciplinary team to test and validate new approaches to the company’s business model, exploring strategic alternatives for expansion and innovation
                \item Contributed to the development of the Nota Quero application, part of the Vestibular Premiado project, focusing on performance and scalability improvements for both the back-end (Node.js with Nest.js) and front-end (React)
                \item Implemented horizontal autoscaling, adjusted Amazon RDS instances, and utilized Amazon SQS queues for messaging, reducing the application's response time from 10 seconds to 0.05 seconds, significantly enhancing the user experience
                \item Acted as a DevOps Associate, bridging the gap between the development squad and the DevOps team, implementing AWS solutions and Infrastructure as Code (IaC)
            \end{highlights}
        \end{onecolentry}


        \vspace{0.2 cm}

        \begin{twocolentry}{
        \textit{São Paulo, Brazil}    
            
        \textit{Mar 2019 to Apr 2019}}
            \textbf{Systems Development Intern}
            
            \textit{Banco Fibra}
        \end{twocolentry}

        \vspace{0.10 cm}
        \begin{onecolentry}
            \begin{highlights}
                \item Conversion of VB6 macros to SSRS 2012
                \item Creation and optimization of SQL queries
                \item Support for the operations team with databases and reports
                \item Development of solutions that improved the quality and speed of existing tools
            \end{highlights}
        \end{onecolentry}



    
    \section{Education}



        
        \begin{twocolentry}{
        \textit{São Paulo, Brazil}    
            
        \textit{Oct 2020 to Dec 2022}}
            \textbf{Faculdade Impacta}

            \textit{BS in Systems Analysis and Development}
        \end{twocolentry}



        \vspace{0.2 cm}

        \begin{twocolentry}{
        \textit{São Paulo, Brazil}    
            
        \textit{Jul 2019 to Sep 2020}}
            \textbf{Universidade Anhembi Morumbi}

            \textit{BS in Systems Analysis and Development}
        \end{twocolentry}



        \vspace{0.2 cm}

        \begin{twocolentry}{
        \textit{São Paulo, Brazil}    
            
        \textit{Jan 2018 to Dec 2018}}
            \textbf{FIAP}

            \textit{BS in Computer Engineering}
        \end{twocolentry}

        \vspace{0.10 cm}
        \begin{onecolentry}
            \begin{highlights}
                \item Health Tech Challenge: In partnership with the Beneficência Portuguesa Hospital, first-year Computer Engineering students were challenged to use new technologies to monitor and reduce noise in hospital corridors by building a line-following robot
            \end{highlights}
        \end{onecolentry}


        \vspace{0.2 cm}

        \begin{twocolentry}{
        \textit{Francisco Morato, São Paulo, Brazil}    
            
        \textit{Jan 2016 to Jun 2017}}
            \textbf{ETEC - Escola Técnica Estadual de São Paulo}

            \textit{TVET in Informatics (Programming)}
        \end{twocolentry}

        \vspace{0.10 cm}
        \begin{onecolentry}
            \begin{highlights}
                \item For my thesis, along with a few classmates, I developed a game using the Unity engine, which was presented at the semester's project fair. The project was a low-poly FPS, and I used ray casting for the computation of gunshots.
            \end{highlights}
        \end{onecolentry}



    
    \section{Projects}



        
        \begin{twocolentry}{
            
            
        \textit{\href{https://github.com/rafifos/dotfiles}{rafifos/dotfiles}}}
            \textbf{Dotfiles}
        \end{twocolentry}

        \vspace{0.10 cm}
        \begin{onecolentry}
            \begin{highlights}
                \item Bootstraped my development environment with a script that installs all the necessary tools
                \item Configured my development environment with Fish, fundle, and custom plugins
                \item Management of multiple configurations for different machines
                \item Private configurations signed with GPG for security
            \end{highlights}
        \end{onecolentry}


        \vspace{0.2 cm}

        \begin{twocolentry}{
            
            
        \textit{\href{https://github.com/rafifos/IosevkaCustom}{rafifos/IosevkaCustom}}}
            \textbf{Iosevka Custom}
        \end{twocolentry}

        \vspace{0.10 cm}
        \begin{onecolentry}
            \begin{highlights}
                \item Customized the Iosevka font with a few ligatures and symbols
                \item Built the font with a custom name and variant
                \item Created a pipeline to build the font with GitHub Actions, with the artifacts available for download
                \item GitHub actions are cached to speed up the build process
            \end{highlights}
        \end{onecolentry}


        \vspace{0.2 cm}

        \begin{twocolentry}{
            
            
        \textit{\href{https://github.com/rafifos/atdownloader}{rafifos/atdownloader}}}
            \textbf{atdownloader (Archived)}
        \end{twocolentry}

        \vspace{0.10 cm}
        \begin{onecolentry}
            \begin{highlights}
                \item CLI tool to download anime episodes from Anime Twist
                \item Written in TypeScript, using oclif
                \item Uses semantic-release to automate versioning, changelog generation and publishing to npm
            \end{highlights}
        \end{onecolentry}


        \vspace{0.2 cm}

        \begin{twocolentry}{
            
            
        \textit{\href{https://github.com/rafifos/com.riotgames.League_of_Legends}{rafifos/com.riotgames.League\_of\_Legends}}}
            \textbf{com.riotgames.League\_of\_Legends (Archived)}
        \end{twocolentry}

        \vspace{0.10 cm}
        \begin{onecolentry}
            \begin{highlights}
                \item Flatpak package for League of Legends
                \item Used Winepak
            \end{highlights}
        \end{onecolentry}


        \vspace{0.2 cm}

        \begin{twocolentry}{
            
            
        \textit{\href{https://github.com/rafifos/leagueoflinux}{rafifos/leagueoflinux}}}
            \textbf{leagueoflinux (Archived)}
        \end{twocolentry}

        \vspace{0.10 cm}
        \begin{onecolentry}
            \begin{highlights}
                \item Snap package for League of Legends
            \end{highlights}
        \end{onecolentry}


        \vspace{0.2 cm}

        \begin{twocolentry}{
            
            
        \textit{\href{https://github.com/rafifos/rom_build}{rafifos/rom\_build}}}
            \textbf{rom\_build (Archived)}
        \end{twocolentry}

        \vspace{0.10 cm}
        \begin{onecolentry}
            \begin{highlights}
                \item Scripts to build ROMs and Kernels for Android devices
                \item Optimized toolschains for ARM
            \end{highlights}
        \end{onecolentry}


        \vspace{0.2 cm}

        \begin{twocolentry}{
            
            
        \textit{\href{https://github.com/rafifos/flyme_device_motorola_ghost}{rafifos/flyme\_device\_motorola\_ghost}}}
            \textbf{flyme\_device\_motorola\_ghost (Archived)}
        \end{twocolentry}

        \vspace{0.10 cm}
        \begin{onecolentry}
            \begin{highlights}
                \item Port of Flyme OS for the Moto X (2013)
                \item Kernel and device tree modifications
                \item Customizations for the Moto X (2013)
            \end{highlights}
        \end{onecolentry}



    
    \section{Technologies}



        
        \begin{onecolentry}
            \textbf{Languages:} HTML, CSS, JavaScript, TypeScript, Shellscript (Bash, Zsh, Fish), Ruby, SQL, JSON, YAML, TOML
        \end{onecolentry}

        \vspace{0.2 cm}

        \begin{onecolentry}
            \textbf{Technologies:} React, Next.js, Node.js, Ruby on Rails, Docker, AWS, Git, GitHub, GitLab, Android OS, Android Studio, Android SDK, Android Emulator, Android Debug Bridge (ADB), Android Virtual Device Manager (AVD)
        \end{onecolentry}


    

\end{document}